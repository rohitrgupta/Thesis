The smart grid is regarded as the future electric grid to provide, low cost, reliable, and continuous supply of electricity. Meeting these objectives requires accurate sensing of grid parameters, and satisfying the quality of service (QoS) requirements of applications. QoS determines the latency within which applications should get required data.  
In this thesis, we discuss the issues of a) improving the data accuracy by identifying and eliminating the errors in the measurements due to inaccurate sensors (biased sensors), and b) satisfying QoS requirements of grid applications by proposing algorithms for the future smart grid. 

