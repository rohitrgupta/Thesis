AS these systems are continuously in the running state, Problems are encountered because of the misuse of the system or wearing of the system because of overload or other environmental conditions. Out of a large number of problems observed in the electrical distribution systems, most common problems could be categorized as follows

\begin{itemize}
\item Short circuit: This type of fault occurs when the supply wire is connected to the neutral wire which results in the flow of a very large current through the distribution wires. There are always some losses in the wiring, Therefore the energy flowing into the system is released by the wires as losses resulting in heating up and eventually catching fire. The wires in any distribution system are not capable of sustaining such a large current for a long duration. To prevent such a huge flow of current, the section of the distribution circuit is disconnected from the rest of the circuit either by the fuses or the circuit breakers.

\item Earthing Fault: When the phase line is somehow connected to the earth or ground, the earth acts as a big sink and the current starts to flow from the phase to the earth. This flow of current also causes the heating up of the wire. Many times it could be because of humans or animals getting in contact with the live wire. Such flow of current could be lethal and should be interrupted immediately. Special relays are installed to detect the occurrence of this event. These relays act immediately and disconnect the impacted circuit from the rest of the system.

\item Circuit Overload: Local electrical distribution systems are like trees where the distribution transformer is the root of the tree and all the appliances are at the leaf nodes. Each branch of the electrical system tree is installed with the maximum current required by the appliances installed under the branch. If the current drawn through the wiring is higher than the maximum capacity it should be because of some unexpected load connected to the system. The losses in the wire are proportional to the square of the current flowing through it. The higher current result in higher distribution losses and heating up of the wire. This heating will damage the insulation and may result in other faults or might cause a fire. To prevent this type of overcurrent, circuit breakers are installed which disconnects the circuit when current beyond a stipulated limit is observed. 

\item Appliance Overload: The appliances installed may start consuming more power because of the increase in the losses because of the aging of the appliances. This will not result in the circuit overload but causes loss of energy and therefore increase the energy bills.

\item Installation Errors: Many times it is observed that the circuit breaker installed is of the wrong capacity. The circuit may be drawing current up to its maximum capacity and installing any new appliance will result in regular overcurrent and tripping of the circuit breaker. Unfortunately, there is no way to identify such boundary conditions in the circuits other than measuring the current at each level in the distribution tree.

\item Failed Appliance: Appliances may fail to perform their desired tasks such as failed lighting equipment or air conditioners which are not cooling. The appliances which can be observed by the occupants are reported to the maintenance staff and are either repaired or replaced. But in many cases, the appliance may go on unobserved and therefore continue to stay in the system. In cases where some service should be performed by multiple parallel appliances like multiple air conditioners are installed to cool a zone, failure of one of the airconditioner will put extra load on other air conditioners but might not impact the occupant's comfort.

\item Underperforming appliances: Many appliances that use mechanical components may decrease in efficiency with time. The appliances that have to perform a predefined task will have to pull more current to compensate for the increase in the losses. This increase in the current is difficult to measure manually and remains undetected. Many such appliances will deteriorate the power factor with time resulting in the same power consumption but more power flowing through the wire. In both of these cases, more power flowing through the wires is not possible to detect manually and the problematic appliance stays in the system.

\item Phase imbalance: Although the electrical distribution system has 3-phases, many appliances are of single-phase type. These single-phase appliances create phase imbalance which is not good for the health of the system. In order to reduce the imbalance caused by the single-phase appliances, the system architect puts an equal load on each phase to maintain balance among phases. New appliances are often added to the system which may increase this phase imbalance. There is no way for the maintenance staff or electricians to check the phase imbalance at each level of the distribution system, therefore, the phase imbalance continues to adversely impact the system.

\item Starting Failures: Many appliances come with their own safety mechanism. For example, when a compressor in an air conditioner, fails to generate enough pressure while starting with its maximum current, the safety circuit shuts down the compressor. Unfortunately, this does not prevent the air conditioner to attempt the start again after a few minutes. Such failed appliances will create a repeated current surge that can damage the system.
\end{itemize}

The problems that are result in the disconnection from the main supply are quickly taken care of and the remaining problems are left unhandled because of the current approach for solving these problems.