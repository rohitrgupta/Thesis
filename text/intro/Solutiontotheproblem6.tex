
\textbf{To identify a sensing method that is generic and does not require the learning of signatures}


A lot of attempts are made to sense the status of the appliance in the electrical systems. With the availability of the smart meters, a more fine-grain power consumption data is available from these electrical systems. It is possible to identify when appliances are switched ON or OFF by analyzing the time series power consumption data collected from these smart meters. With the introduction of IoT devices, it is now possible to install a large number of sensors and colled data from them. This enabled the utilization of a verity of sensors in collecting the data from these sensors. 

Unfortunately, all effort is towards the measurement of the energy consumed by the appliance. These approaches use some behavior of the appliance to identify if the appliance is on or off and compute the energy consumed by the state of the appliance. Most of the techniques are highly dependent on the signatures learned to identify the appliance state. Many of the technique requires learning of the signature of each appliance or on-site learning which makes them less suitable for large scale deployment.

These techniques fail to identify the state of the appliance in many situations like change in the signature, change in location, presence of multiple appliances, etc. Therefore, the information made available by these techniques does not provide any insight into the health or the performance of the system. A technique that does not require learning the signature of each appliance is more suitable for large scale deployments.

A well-known observation of variation in voltage observed at the load level can be utilized to identify the load. Unfortunately, there is a large variation in the supply voltage and a large number of variations are observed in the voltage because of the load variation in the rest of the system. This presents a big challenge of removing noise to correctly identify the load state of the subtree. The method of identification of removal of noise is devised to utilize the voltage sensed at various locations. 

In this work, we have introduces the technique to use voltage values sensed at different locations in the system and spatial variation in the voltage to determine the state of the system, incorrect system configurations. The system state information together with the power consumption measured in the mains meter is utilized to determine the load consumption of each section of the system. The temporal variation in the voltage is utilized to determine the correct layout of the system. This information is required to rectify the phase imbalance situation. Temporal variation in voltage is also used to detect current spikes to determine the location of the failed appliance. As voltage can be sensed by plugging a device into the socket, the task of layout identification and spike detection can be performed easily with minimal equipment and effort.

We present a voltage sensing technique that can be installed without learning the signature of the appliance. Voltage variation is a generic effect caused by each appliance, therefore, it can be employed for identification of the state of any appliance. It can also be used to identify the zone level power consumption which is useful for the identification of the location of the problem in the system.

