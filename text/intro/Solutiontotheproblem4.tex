\textbf{To design algorithms to identify problems in the system}

There are multiple problems that can be present in the distribution systems in the buildings. These problems could be of continuous overload because of unwanted or underperforming appliances. They could be sudden short-term spikes of current because of faulty appliances or unbalanced load in the system because of the inappropriate setup of the appliance and instrumentation.  

These problems mostly go undetected by the maintenance staff, therefore, no corrective actions are taken to eliminate them. As the total power consumption is already measured by a smart meter at a higher sampling rate, it is possible to identify some unexpected behavior in the system using only the smart meter data. 

In this work, we present the algorithm to identify the spikes in the mains meter data and thus identify the presence of faulty appliance in the system. With the technique to compute zone level energy and using the voltage sensing technique, it is possible to find the source of the spike and unexpected changes in the load profile of the zone to detect if the source of the problem is present in the zone. 

Using the voltage sensed at multiple places we have also developed the mechanism to identify the phase of individual sockets thus enabling the correction of the phase of the wrongly labled smart meters or lines and improving the overall phase imbalance of the system.