The maintenance of the electrical system is generally done only when the system break downs. This is because of multiple reasons. The electricity should be made available without interruptions. Maintenance of the distribution may require either operation in the live system or shutdown of the system. Until the exact health of the system is known, it is difficult to identify the rectifications required in the system. Because of this preventive maintenance are less common for the electrical distribution systems. 

The problems mentioned in the above section can be detected if the power consumption of each appliance is monitor. IT is straightforward to identify the problematic section or appliance based on the trend of power consumption. But it is very costly to continuously measure the power consumed by each appliance. It is also not practical to measure the power consumption of each appliance at a regular interval. With the introduction of thermostat based appliance and intelligent appliances, it is also not possible to identify the overall power consumption behavior of the appliance by measuring the power consumption in one measurement. 

With the invention of smart meters that can sense and report power consumption at a higher sampling rate, it became possible to observe the power consumption and derive various inferences from the power consumption pattern. It is now possible to identify how much power is demanded by a particular appliance and when.

Each appliance can produce different power consumption patterns which can be observed to identify ON and OFF appliance by just observing the power in the mains meter data. Thus it is also possible to identify the variation of power consumption pattern of an individual appliance. This introduced a new field of research to compute the power consumption of individual appliances from the mains meter data.

This Idea of identification of appliance state and thereby computing the power consumption was first utilized by the Hart. The active and reactive power consumption pattern was used to both identify the status of the appliance and compute power consumed by the appliance. This field of None Intrusive appliance load monitoring (NIALM) and a number of sensing techniques and algorithms are tried to achieve this goal. The focus of this field of research is to measure the power consumption of individual appliances by minimal sensing. Unfortunately, NILAM depends on expected behavior of the appliances and fails to work when the appliance misbehaves or some other unexpected elements are introduced into the system.