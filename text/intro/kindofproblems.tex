% of electrical systems and faults 

Electricity is one of the most utilized form of energy. It is popular because of the ease of distribution and on-demand availability. The electrical distribution systems are efficient, convenient, safe, and environment-friendly compared to all other energy sources. Unfortunately, it is not easy to store and carry electricity around. , Therefore it should be produced and distributed at the time of its demand. The electricity production and distribution system should always be in the ready state to cater to the everchanging demand. 



Electricity distribution systems existed for a long time. These systems are time tested and are very stable in delivering the power to the individual appliance consistently. They are very reliable and modern life is highly dependent on these electrical distribution systems. Unfortunately, these electrical distribution systems are without any backup and are always in use. So they rarely get downtime for preventive maintenance or health monitoring. Therefore they are mostly maintained in case of breakdown.

There are mechanisms built into electrical distribution systems to handle Critical errors that should be handled in real-time to prevent the entire system from breakdown and accidents. Circuit breakers are the equipment that identifies such a critical fault and disconnects the faulty part form the rest of the system. These circuit breakers are installed at the strategic location to isolate the faulty section and restrict the impact of the faults on the rest of the system. Some circuit breakers are also designed to minimize the impact of accidents on the occupants caused by faults or improper handling of the system. Such circuit breakers are commonly installed in residential and commercial spaces.

The circuit breakers are disruptive in nature and stops all the operations in the faulty section. Therefore they are designed to prevent only the most severe and catastrophic faults in the electrical systems. Less severe problems and losses in the electrical distribution systems are tolerated in order to maintain the continuous operation of these systems. These losses are considered insignificant relative to the efforts required to mitigate such losses. These losses are very small and insignificant at the household and building level but are huge at the global level.

