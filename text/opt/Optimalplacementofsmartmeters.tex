The default sensor for energy measurement is the smart meter because it is the only sensor which directly provides the power consumed. In the following analysis, we studied the number of smart metes required to identify the state of all the appliances. Our simulations show that the number of smart meters required is directly proportional to the number of appliances installed. The analysis of the diversity of the appliances revels that the number of smart meters required is dependent on the diverse nature of the appliances. If the installed appliances are of similar or identical electrical nature, then the number of sensors required is very near to the total number of the installed appliances, but if the power consumption of the installed appliances is diverse, then the number of meters required can be much lower.

Figure \ref{fig:optsens} shows that the number of smart meters required increases with the appliance installed. Initially the ratio of the number of the smart meters to the number of appliance decreases, this is because the number of sensors required when only one appliance is installed could not be less than one. As the number of appliance increases the ratio decreases and becomes constant. It can be observed in the figure \ref{fig:sim_V_b} that when number of appliances is above 4 this ratio remains constant. This behavior is same for all types of the layouts. In the case of the $type 2$ layouts the number of smart meters required is around 60\% of the number of appliances installed. Therefore we can assume that merely using the algorithm to strategically install the smart meters, the number of smart meter requirement can be reduced by 40\%.
