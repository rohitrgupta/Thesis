In this strategy, similar to installing the current sensors, only one smart meter is installed at the mains and voltage sensors are installed at all other places. Instead of using the smart meter if the voltage sensor  are installed to detect the drop in the voltage, the number of sensors required is similar to the number of smart meters. Our understanding for this observation is because the smart meters (after installing the mains smart meter) are not adding much information by providing the power measurement. Therefore the only additional information obtained from the meters is the voltage droop.

For the $type 2$ load, the number of sensors required is slightly higher than the smart meter with voltage inference, but it is significantly less than the number of smart meters required when the voltage inference is not used.

Therefore  the number of sensors required for this case
    $$O_P \le  O_V$$
 Overall the cost of the voltage sensor is much less than the smart metre

 Although the number of sensor is almost similar the cost of sensors is much less

