The problem of the identification of the sensor requirement is Np-Hard.  In order to compare the performance of different algorithm we ran the different algorithms for the same set of layouts and compared the number of sensor requirement and the time taken by the algorithm to for find the sensors required.

The number of the system state is the Cartesian product of the individual appliance state. Let $h$ be the average number of state for the appliances, and $m$ be the number of appliances. The number of system states is of the order of $O(h^m)$. To check if the sensed values are distinct the values sensed for each state should be compared at least once. Therefore the time complexity of checking one column in the sensor values table will be $O(h^m)$. The sensor table will consist of $n$ columns where $n$ is the number of possible sensors. For finding the optimal placement using brute force algorithm first each column is checked individually than all the combination of two columns are checked and it is continued till a suitable placement is found. So the number of iterations required is:
$$^ nC_1 + ^nC_2 + ^nC_3 ...  = 2^n $$
Therefore in the worst case the total number of operations required for finding the optimal by brute force in $O(h^m * 2^n)$. assuming that any appliance will have at least two states, $h$ will be $2$ in the best case.  Also as any electrical parameter can be measures at each appliance the number of possible sensors will be at least equal to the number of the appliances installed. Therefore, in the best case, the time complicity for finding the optimal by brute force will be  $O({(2^n})^2)$. for just 16 appliances the required operation will be more than a billion. In our simulations the time required for finding the optimal in such case was  several hours. As the number of appliance in any household may be much higher than 16, algorithm is impractical for regular use.

The brute force algorithm will always provide the optimal solution for the problem but is costly because each combination of the sensor should be checked. When the appliances are diverse, the number of sensor requirement is very low and the algorithm can find out the optimal placement in relatively less time. The sensor requirement increases when the appliances are less diverse and will be very high if all the appliances installed are identical in power consumption. When the diversity of the appliance is less, the brute force algorithm takes longer to find the optimal solution. Figure \ref{fig:opttime} shows that the time required for finding the optimal increases exponentially for all types of layouts but it grows slowly for the type 2 and type 3 layouts.

Almost similar performance is observed irrespective of the type of the sensors selection.
