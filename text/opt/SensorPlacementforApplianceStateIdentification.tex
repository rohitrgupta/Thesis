Although many approaches are available to track appliance operations, most of them are not suitable for office like buildings where a large number of similar appliances are usually installed. All instances of the same appliance are likely to have similar power consumption as well as high-frequency characteristics.
Installing additional sensors is the only option available in such cases. We  present an algorithm for the placement of voltage and current sensors and get full visibility of the state (i.e., ON/OFF) of each appliance in a circuit.

Consider the sample electrical layout, shown in the Figure \ref{fig:distribution}, used to demonstrate our approach.
%rg
The power consumption values of the appliances are selected to point out different desegregation difficulties and explore their solutions.
The power consumed by the three appliances is given in Table \ref{tab:powerRatings}. Appliances 1 and 2 have different power signatures, therefore it is straightforward to disaggregate the consumption if only these two appliances are present. It becomes difficult when overlapping appliance signatures are present, for example, appliances 2 and 3 both can cause an increase of 2x power. Given a transition of 2x Watt increase in power consumption, it will be difficult to determine which appliance is causing it. Some disambiguation can be achieved if the state from which the transition occurs is also included in the decision making, For example if the system is in state O,O,A and a 2x increase in power is observed, it can only because of appliance 2, but not all instances of the problem can be resolved using such information.

For example, when the power consumption increases by x Watt, it may not be possible to determine whether the increase is because of appliance 1 or appliance 3.
%kk  3,  not 1, no? yes
Figure \ref{fig:sys} shows all the possible states of the system based on the state of each appliance and possible change of appliance states when only one appliance is changing its state.
Each state in the diagram is associated with a tuple, ($A,O,B$), where \textit{A, O} and \textit{B} respectively representing the state of the three appliances: appliance 1 is in state $A$ consuming x Watt, appliance 2 is in state $O$ consuming 0 Watt, and appliance 3 is in state $B$ consuming 2x Watt power. If the power consumption of the system decreases by x Watt, the next possible state of the system can be ($O,O,B$) or ($A,O,A$). Such non-deterministic transitions are shown by dotted lines in this figure.
In such a situation, a probabilistic model can be used which utilizes other information like appliance behavior to resolve such confusion. For example, if appliance 1 is a bulb it is less probable to be switched ON at day and more probable to be switched ON at night. Appliances like Air conditioners consume power in cycles. The time period of the ON-OFF cycles of such appliances can also be used to identify their operation. The probability of use of different appliances will be different at different times. The probabilistic algorithm can utilize this probability to identify the correct appliance. But with a large number of appliances with  similar probability, such algorithms will not give accurate results.
