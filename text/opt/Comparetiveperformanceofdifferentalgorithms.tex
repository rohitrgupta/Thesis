In order to evaluate the performance of different algorithms to find the optimal placement of the sensors, and to evaluate the number of each type of sensor required, tests were conducted on a large number of electrical layouts which were generated by using a variety of appliances. Following three types of layouts were generated

\begin{itemize}
    \item Type 1: Mostly similar appliance
    \item Type 2: Moderately diverse appliances
    \item Type 3: Highly diverse appliances
\end{itemize}

For the generation of the $type 1$ layouts, the number of allowed appliances were restricted to a few (2-3) depending on the size of the layout. The selection of the appliance for the $type 1$ layout was such that the power consumption of the appliances was identical for some of the states. Selecting power consumption in this way result in same power consumption for multiple states of the system and a large number of confusions are observed in the system states. For the $type 2$ layouts, the number of the allowed appliance was increased in the range of A1-A2 and for generating the $type 3$  layouts the number of allowed appliance were further increased.

For all the layouts, the length of the wire from the distribution boards to the switchboards and from switchboards to the appliances was taken in the range of $10-50$ meters, and the number of appliances connected to the switchboard was taken in the range of $2-8$ which are the commonly observed parameters in any electrical layout.  Multiple level of the switchboards were also allowed.

The total power consumption for all the appliances under a switchboard was used to identify the current rating and gauge of the wire to connect the switchboard to the distribution board. The wire of  higher diameter (lower gauge) than the recommended current consumption were used.   The electrical parameters of the copper wire for the selected gauge and the length of the wire were used to compute the resistance of the wire and the expected voltage drop at the appliance and the distribution board.

