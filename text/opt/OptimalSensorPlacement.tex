We now examine the following problem: identifying the number, type, and location of each sensor required for determining all the unique states of the system. A simple approach to make sure to identify all system the states is to install a sensor at each appliance or zone. The number of sensors needed in this case will be equal the number of appliances.

We show now that the objective can be achieved by installing a smaller number of sensors by solving it as a linear optimization problem as follows.

Let there be an electrical system than can be divided into $n$ zones represented by a set $$Z = \{z1, z2, z3 ... zn \}$$. Power consuption of zone can be represented by a set $$P_{zi} = \{p_{z1}.1, p_{z1}.2 ... p_{z1}.m1\}$$. Here zone one can consume one of the $m1$ possible power consumption possible. A zone can consume a power out of a finite number of significant power level. The total number of states in the system will be given by 

\[
|P|  = \prod_{i = 1}^{n} |P_{z1}|  
\]


Our example circuit will have 12 distinct system states, 12 being the product of the number of unique states of each of the appliances.

We define the state of the system by a set of tuples 

$$T = \{(p_1(t), p_2(t),...p_n(t))| p_1(t) \in P_{z1} , p_2(t) \in P_{z2} ... \}$$ 

where each element of the tuple is the power consumed by a zone.

In the system it is possible to install sensors at various places. Let these places be defined by the set $$L = \{l_1, l_2,...l_o\}$$

at these places a number of parameters can be observed. These parameters are represented by a set $$F = \{f_1, f_2,...f_r\}$$

Therefore the possible sensors that can be installed to observe different parameters can be given by set $$S = \{(l_i, f_j)| l_i \in L, f_j \in F\}$$ 

Therefore the number of possible sensors is given by $|S| = |L| * |F|$

Our objection is to select the minimum number sensors that when installed will provide sufficient information to identify the state of the entire system.

In our example, the possible places where the sensors can be installed are 

${Mains, Aplliance 1, Aplliance 2, Aplliance 3, SB 1, SB 2}$
We are selecting only three type of sensors for this example namely 
$$Power, Voltage, Current$$
Here the power is sensed by an smart meter. As a smart meter will also profide voltage and current values, the voltage and a current values provide by the smart meter is but it is not counted as a seperated sensor. It is also assumed that a smart meter is installed by default at the mains. 

The short notations used by different sensors are as follows. Current sensor at switchboard is 
$$i_{s1}$$ 
the current at the appliances are 
$$i_{a1} , i_{a2}i_{a3}$$, 
and the voltage at the appliances are 

$$\Delta V_{a1}, \Delta V_{a2}, \Delta V_{a3}$$

We define a decision variable $u_j$ as
\begin{equation}
u_{j} = \begin{cases} 1,& \text{if sensor } j \text{ is used } \\
0, & \text{otherwise} \end{cases}
\end{equation}
The value sensed by sensor $j$ in system state $i$ is given by $x_{ij}$.
%When the sensor is used, all the values sensed by it will be available for identification of the system state.
The objective is to
$$\min \sum u_j$$
such that we obtain a distinct set of values for the system states, i.e.,
$$|set(x_{ij} * u_j)| = m$$
The problem is NP-hard as all the possible sensor placements should be checked for finding an optimal solution, making the time complexity of searching for the optimal solution, $O(2^n)$.
