In large buildings where a lot of similar appliances are installed, disaggregation of the appliance load is a challenging task. Utilization of already installed instruments can be a big factor in deciding the practical feasibility and success of such tasks.
Both voltage and current sensing have their own advantages in different places. Current sensing can be easy if no sockets are available and phase wires are accessible separately to install a clamp on current transformers. Voltage sensing, on the other hand, can reduce time and cost of installation or it can even eliminate by utilizing the voltage sensed by the already installed smart meters. It can also provide ad-hoc measurements through sockets for quick fault localization.

In this work, we exploit the fact that change in an appliance's power consumption will result in a change in the current flow and voltage drop at different places in the electrical system. Simple sensors can be installed to measure these changes and identify the appliance status. We have also presented a novel technique that uses sensed voltage and utilizes it to improve disaggregation results. We also presented an algorithm to decide about the placement of these sensors to yield maximum insight into the systems.

During this process, we experienced a lot of data collection challenges which we overcame using the sensed voltage. We have developed a robust sensor data logging system capable of automatic configuration and correction of time stamping and phase labeling errors.  The latter problem is encountered quite often in practice but has not been explicitly studied for buildings.  For example,  when we install solar roof-tops both in residences and in larger buildings, we need to ensure that a panel is connected to the right phase. In fact, most building managers have an incorrect circuit map of the building which can be quite large \& complex in commercial buildings. So identifying phase at every point and socket correctly is an important aspect of correcting the circuit map of a large building. Since buildings are also commercially billed, they are penalized for imbalanced phases by the utility that one cannot address without identifying the phase of different loads in the building.

We have made a strong case for voltage sensing which is capable of providing a lot of insights into the system. We showed that already installed hardware along with some additional voltage sensors, can be utilized to gain more insight into the system, one such insight leads to effective fault localization. These techniques can provide even better results by utilizing more precise voltage sensors.
