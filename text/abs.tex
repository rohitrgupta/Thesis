A large number of electrical appliances are installed in buildings. With time, more appliances are added and old appliances and replaced. The dynamic nature of the electrical system in a building results in a variation in the age of the appliances. These appliances are bound to degrade with time and fail eventually. Once appliances are installed there is no way to automatically monitor the health of these appliances or to detect malfunctioning appliances and perform corrective action.

The current practice adopted by the maintenance staff is to maintain the blueprint of the electrical layout and refer to it when a problem is reported. Major components of the electrical system are hidden and can only make sense when correct blueprints are referred. The electrical layout in the building may change with time resulting in a change in the consumption pattern of each section of the building. It becomes difficult to keep on updating the electrical layout, therefore, it could be incorrect many times.

When a problem is reported, the maintenance staff measures the current flowing through different cables to identify the location and nature of the problem. It is not practical and economical to monitor various electrical parameters at all locations in the system to detect such problems. Therefore the maintenance is done mostly in a reactive manner. The failed appliances are replaced or repaired only when such incidents are reported. The current mechanism to identify these failure and performance loss depends on occupants reporting such incidents and is reactive and proactive.

As there is no easy way to monitor the health of the electrical system, the malfunctioning appliances which are not observed by the occupants, stay in the system and keep on wasting power. They wast energy and also put extra load on the system which also decreases the life of the system.
