Many disaggregation techniques use high-frequency voltage and current data. Our meter is capable of collecting and providing 9.6 kHz voltage, current, and power data which can be used by many such techniques. We have collected such data for a variety of appliances.

Figure \ref{fig:hfCurrent} shows the high-frequency current data collected for four different types of appliances which is iron, fridge, a microwave, and a laptop. The current drawn by the iron is almost identical to the voltage waveform. this indicates that the iron is a pure resistive type of load.
The current drawn by the laptop is remains zero for the initial part of the cycle and shoots up at a certain input voltage. This is the typical characteristic of the AC to DC converters. We expected the current drawn from the fridge to be a shifted sinusoidal but it has some unexpected shape. On investigation, we found out this is because of the third harmonic. The current waveform of the microwave is also as expected.

Figure \ref{fig:hfVI} shows the VI-trajectory drawn by plotting v on the x-axis and current on the y-axis for the same four appliances. The VI-trajectory for the iron is almost a straight line which also indicates that the iron is a pure resistive type of load. The current VI-trajectory for the laptop has a sudden jump at both the highest voltage values. This is the typical characteristic of the AC to DC converters. The VI-trajectory for the fridge is a distorted oval which indicates that it is an inductive load. The VI-trajectory for the microwave is a galaxy like shape as expected.

