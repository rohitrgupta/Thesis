An energy meter senses the consumption of power for every cycle. Data collected for the duration of one second is then aggregated to obtain the power consumption for the second. This data can be queried from the Industrial meters using the MODBUS or similar such protocols. When per-second data is collected, the data volume is too high for large scale deployment. The change in power consumption is less frequent as the state of an appliance does not change very frequently. Smart meters installed in homes send the data at a much lower rate at an interval of 5-15 minutes. When the energy data is sent at such a lower rate all the information about the transients is missed out. As our meter holds the power consumption for every second and it relays the information whenever the changes in power consumption are significant enough. The number of messages required to send the change in power together with the 15-minute interval energy reading is significantly lower than the messages sent every second. We analyzed one-month data from 50 homes to evaluate our messaging strategy. We found that 3000 messages are sufficient to rebuild the per-second level power consumption details for 91\% of the cases. Figure \ref{fig:compressRish} shows the power consumption profile of one of the houses. The first graph is drawn using 1250 samples while the second one is drawn using only 63 samples. There was no significant difference between the two profiles. We can achieve this reduction in the data transfer because the communication is initiated by the meter and not by the master.

