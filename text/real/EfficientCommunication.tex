Industrial meters like Schneider EM6400 \cite{EM6400MultifunctionMeter} use the slow Modbus interface to transfer the power consumption data. Modbus protocol works over the RS485 serial bus which is a half-duplex and supports slow data transfer rate typically 19.2 Kbps. All the communications on the RS485 bus are initiated by a master. I smart meter case the master sends a  request and the meter responds by sending the latest readings back.

In Modbus communication, the master is unaware of the state of the meter and does not know when the meter reading has changed. When data is required to be collected at 1 Hz rate, which is the same as the meter's refresh rate, the master should request the dare at a faster rate to avoid looking some intermediate meter readings. If identical reading is received such readings are discarded. The Modbus communication is initiated by the master, therefore, it is efficient for the regular interval query which is slower than the data refresh rate but is highly inefficient in other cases.

When the objective of installing the meter is to identify maximum and minimum power consumption during a period for the purpose of billing or to identify the time when the power consumption has changed for the purpose of disaggregation, it is highly inefficient to read power consumption from the meter every second. As the meter can compute such information very easily, changing the way in which data reading is performed can drastically reduce the communication overhead.

RS485 and thus Modbus can support up to 255 devices on a single bus but this number reduces drastically when data is collected at 1 Hz frequency. In Modbus protocol, when the power data is collected at the same rate as data refresh rate the master should query the meter with a faster rate than the data refresh rate to avoid missed data and discard the duplicate data received. This causes additional communication overhead and the communication channel is 100\% utilized when as few as 5-6 meters are connected on a single RS485 bus.

Our meter is capable of detecting the change in different parameters sensed and initiate the high frequency and the low-frequency communication based on these events. The possible events that can be used in the meter are threshold based change event and timer based events. These events can be used to collect the 1Hz or lower frequency power data or high-frequency current, voltage, and power samples.


