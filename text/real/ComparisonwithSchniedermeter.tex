Our lab has three parallel supplies for plugs, fittings like lights and fans, and for air conditioners. We have installed the Schneider EM6400 meters to monitor and collect power consumption data at a frequency of one Hz. This data is collected by a raspberry-pi and sent to MQTT server from where different applications like database logger or data analyzer to subscribe to this data. We have installed our meters in In parallel with the Schneider EM6400 meter that collects the data for the fittings. We have selected this meter because we can easily get downtime and a number of different available power consumption levels observed in this meter.

We collected data from our meters at the rate of 1 sample per second and sent to the MQTT server similar to the Schneider meter. We compared the data for a number of days and analyzed the difference between the voltage, current, power, and frequency observed by the two meters. Figure \ref{fig:fvcpcompare} shows the comparison of the four parameters observed by the two meters for a period of one day. Table \ref{tab:rmsError} shows that the average of the absolute error is very low for all the parameters. The parameters observed by the two masters are almost identical except for occasional slight differences. Explanation of the observations is as follows.

Frequency:  No significant difference was observed in the frequency measured by the two meters.

Current: The current measured by the two meters is almost always the same except for the cases except for very few samples. This difference is only observed when some appliance is switched ON or OFF. Sometimes there is a delay in the Schnieder meter reporting the new current value. One of the reason could be because of the delay in querying the readings from through the MODBUS protocol. As the raspberry-pi adds the timestamp to the data when it is received from the meter, the timestamp is always delayed by a few hundred milliseconds.

Active Power: The active power measured by the two meters is almost identical except for a few cases similar to the current measurement. Active power is not a product of RMS voltage and current but is also a function of the power factor of the load. Our load of fans is a reactive load which indicates that the measurement of active power is accurate in case of such loads also.

Voltage: On the larger range the voltage change in the same range but there is always a slight difference in the readings collected from the two meters. It is observed that reading of one meter follows the reading of another meter. The sampling in the two meters does not start at the same cycle therefore even if the two meters are measuring the same value the average taken across 50/60 different cycles will be slightly different. As the voltage is continuously changing the values measured by any two meters will always differ slightly.
