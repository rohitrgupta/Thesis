As a micro-controllers can only perform ADC for the voltage, the current signal should be converted to the equivalent voltage signal to make the measurement. Following are the common current measurement methods available

\begin{description}
\item [Series resistance:] Voltage across a precision series resistance is measured to sense the current flowing through it. It is a very popular current measuring technique but it is an intrusive measurement technique which does not provide the isolation of the circuit.

\item [Hall effect sensors:] These sensors are smallest in size, they use the electric field generated by the flowing current through the sensor. This requires the current to be passed through the sensor which is done by cutting the wire and connecting the sensor in series with the load. We have avoided the Hall effect sensor for ease of installation.

\item [Sensor modules:] There are some sensors modules like WL1500 available in the market which provide the voltage output after adding the bias.  %These sensor modules make the circuit smaller but reduce the flexibility of measurement range, therefore, separate modules are required when changing the range of measurement of the smart meter.
These sensors are slightly costlier compared to the other sensors, we are not using this kind of sensor to reduce the cost of the meter.

\item [Current transformers:] Current transformers are the most commonly used current sensing method. they need additional circuits for adding the bias to the sensed signal. They are very accurate, and low in cost which makes them most suitable for the smart meters. The advantages of these current transformers are that by changing a single component (burden resistor) or by just changing the turn ratio of the transformer, the sensitivity of the measurement can be changed.  In our design, we are using the current transformer and a bias adding circuit using a few components as shown in figure \ref{fig:currentCircuit}. This reduces the cost while providing the flexibility of measuring range.

\end{description}

