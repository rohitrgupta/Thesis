Smart meters are the key infrastructure required for the implementation of smart grids that can enable the realization of tasks like demand response, efficient power quality management and sharing of resources like battery and solar PV. Existing smart meters are designed for the purpose of collecting energy consumption data only. These meters will not be able to meet the new requirements arising from the smart grid implementation. Considering these changes in the smart grid, the smart meter should be a combination of modular and upgradeable hardware and software. It should also enable researchers to collect data in the desired way. To enable exponential growth in the metering infrastructure, the following capabilities are desirable.
`
% Task performed by smart meters include tasks like state-of-the-art measurement and calculation, hardware calibration, and communication.

%Knowledge of peak or off-peak periods, energy usage patterns, two-way communication between end-user and utilities is beneficial for utilities. Metering through smart meters in a smart grid can provide these abilities, but methods to implement these capabilities are limited today. Smart meters include measurement and calculation section, hardware and software calibration, and communication capabilities. However, current industrial and research purpose meters fail to adapt to the ever-changing technology. They do not provide enough analytics to observe and understand the usage patterns. Also, they are costly and bulky in design and hence, still less prevalent in developing countries like India.

