The most commonly used technique for voltage level conversion in alternating current systems is the transformer. Transformers are used everywhere in the electrical system to change the level of the voltage. The transformers are bulky as they are designed for high power transfer. They are not designed for accuracy and work efficiently only when the amount of power transferred is high. They also suffer from hysteresis losses which makes them less reliable for measurement purposes. The transformers are made by ferrite layers and they vibrate as the polarity of the induced magnet changes with each cycle of the supply. This continuous vibration results in a lot of noise. Because of these limitations transformers are not suitable for reducing the voltage for sensing purpose.  There are instrumentation transformers but are too costly to be used in a low-cost smart meter.

Some very low-cost voltage meters use capacitor charging and discharging circuit to measure RMS voltage directly. This technique is not suitable when power is needed to be measured because power computation requires instantaneous measurement of both voltage and current.

There is rarely any low-cost sensor available to sense the high voltage and current. In the absence of such sensors, the most popular option used for voltage sensing is to reduce the voltage by using the resistive voltage divider circuit. This circuit is very small and economical but it does not provide electrical isolation.  The output of this circuit has both positive and negative magnitude. To take care of both of these issues, we have used an op-amp circuit as shown in figure \ref{fig:voltageCircuit}.
An op-amp is used to convert the voltage within the sensing range of the micro-controller and to provide isolation to the microcontroller.

Measurement of the electrical attributes should be done continuously and the end of one cycle should be treated as the beginning of the next cycle. Identification of the change of voltage from a negative value to a positive value is treated as a change of cycle in this case. We have used another op-amp to generate a square wave from the input signal that is used to trigger the processing of the RMS computation for the cycle.
