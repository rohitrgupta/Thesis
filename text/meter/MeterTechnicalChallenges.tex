A lot of research is done in the field of energy disaggregation but the biggest difficulty in bringing this technology out of the laboratories is the lack of low-cost data collection equipment.  The technology to build the meters capable of disaggregation of the energy consumption is already around but such meters are not available in the market as a product. The biggest obstruction in making such a product is the lack of field research because of the unavailability of hardware that can provide the platform for such research. Majority of the research is done on the data collected by the available infrastructure or by installing the high-end industrial equipment.

Most advanced metering infrastructure can provide the aggregated power consumption data up to a few minute granularities. Data collected by at such frequency is sufficient for utility for its different purposes. Benefits of data collecting at a higher frequency do not justify the increase in the cost of collecting transmitting and processing the increased volume of data. In all digital energy meters, the power consumption data is collected for each cycle but only the aggregated data is provided by the meters. The information that can be useful for energy disaggregation is processed by the meter but is not made available to the outside world.  These meters are made from the components that do not access to their internal data, therefore, it is not possible to add this feature to the existing meters.
