The naive approach to finding the consumption of each appliance is to measure the power at each appliance. There are multiple challenges involved in measuring each appliance.

\begin{itemize}
\item \textbf{Technical skills required to install a meter:} It is not easy to install a smart meter, it requires a set of precise connection, may involve working with live high voltage wires or a local blackout, and need a decent amount of time to install test and make the meter data available to the system. It is not a plug and play devices.
 \item \textbf{Not all the appliance can be metered:} Many appliances are plug level devices such as laptops. To measure the power consumption of such devices a meter should be attached to the device rather than on the wall mount which will be a cumbersome option.
 \item \textbf{Losing aesthetic sense of the building:} meters are bulky, they need big space to be installed and mounted on for safety reasons. installing a meter on each wall in each room will be visually unattractive and space will lose its aesthetic sense.
 \item \textbf{Huge volume of data to be collected:}, As the current day smart meter provides only the instantaneous power reading, each meter installed will generate massive data that should be communicated and processed at the server. This will require reasonably large communication bandwidth, storage capacity, and processing power to analyze the data.
 \item \textbf{Too much cost involved:} Each meter installation will require at least 10000 Rs of expenditure, the benefits of reducing the energy consumption and reducing the carbon footprint are not enough to motivate customers or the utility to invest such a hefty sum.
\end{itemize}


