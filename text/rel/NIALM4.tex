High-frequency current sensing is often utilized \cite{patelFlickSwitchDetecting2007}, even in commercial products \cite{SmartBuildingManagement}. These identify appliances based on the variation in current during a single voltage cycle. The current waveform is different for different appliances but it is difficult to identify appliances when the current for more than one appliance is combined. Techniques like binary factorization \cite{langeBOLTEnergyDisaggregation2016} can be used to identify multiple appliances, but the technique is limited to two concurrently running appliances and thus requires sensor placement near the appliances.
High-frequency current and voltage can also be sensed simultaneously for a single cycle \cite{hassanEmpiricalInvestigationVI2014} to plot its path in the voltage and current domain also called VI-trajectory. The features of this trajectory are learned as the appliance signature, but this technique is also limited to identifying one appliance at a time.

Current sensing requires accessing the electrical wire supplying power to the appliance or through a smart plug; this may not always be feasible. Voltage sensing in comparison is eased by the fact that it can be sensed at any nearby place. For example, in a household, installing a current sensor for an air conditioner will require either accessing the wire to install a clamp-on sensor, or provide supply to the Air conditioner through a smart plug. These options will be difficult or may not always be feasible. On the other hand, voltage sensing is possible through the existing plugs.
