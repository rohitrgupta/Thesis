Confusion in appliance identification is a common problem, and various attempts have been made to solve it. Disaggregation approaches developed to address this problem differ in the sensing mechanisms or in the algorithms they use, or both.

High-frequency current sensing is often utilized \cite{patelFlickSwitchDetecting2007}, even in commercial products \cite{SmartBuildingManagement}. These identify appliances based on the variation in current during a single voltage cycle. The current waveform is different for different appliances but it is difficult to identify appliances when the current for more than one appliance is combined. Techniques like binary factorization \cite{langeBOLTEnergyDisaggregation2016} can be used to identify multiple appliances, but the technique is limited to two concurrently running appliances and thus requires sensor placement near the appliances.
High-frequency current and voltage can also be sensed simultaneously for a single cycle \cite{hassanEmpiricalInvestigationVI2014} to plot its path in the voltage and current domain also called VI-trajectory. The features of this trajectory are learned as the appliance signature, but this technique is also limited to identifying one appliance at a time.

Current sensing requires accessing the electrical wire supplying power to the appliance or through a smart plug; this may not always be feasible. Voltage sensing in comparison is eased by the fact that it can be sensed at any nearby place. For example, in a household, installing a current sensor for an air conditioner will require either accessing the wire to install a clamp-on sensor, or provide supply to the Air conditioner through a smart plug. These options will be difficult or may not always be feasible. On the other hand, voltage sensing is possible through the existing plugs.

High-frequency voltage sensing is also utilized to measure high-frequency electromagnetic interference (EMI) noise generated by various switching events happening in the electrical circuits and is used for both mechanical \cite{guptaElectriSenseSinglepointSensing2010} and electronic switches \cite{chenWhatCanWe2015} or even to identify the continuous noise generated by SMPS and motors. All the above approaches use high-frequency sensing that requires installing special equipment to sense the huge amount of data and process it locally. This results in a high cost of equipment.

EMI measurement determines the distance of the appliance based on the intensity of the sensed noise. Our approach also utilizes the intensity, i.e., magnitude, of the voltage change to determine the location of the appliance in the electrical system. However, our approach uses low-frequency voltage measurements which do not require specialized equipment as in the EMI sensing. Since all the appliances are electrically connected, EMI sensing will sense all such appliances even if not intended.
Voltage difference will be observable only when the current drawn by the appliance is flowing through the wire across which the voltage is measured. This enables voltage sensing to monitor nearby appliances but is not impacted by other appliances which are supplied through separate lines. This reduces the noise caused by distant appliance thereby limiting the scope of sensing to just the targeted appliance.

Other sensing techniques involve sensing sound, light, vibration or change in temperature caused by the appliances \cite{hnatHitchhikerGuideSuccessful2011}. These techniques are difficult to use in practice because a high skill set is required to install these sensors, which can make it time-consuming and or expensive to install.

Powerline parameters have been used in power grids for a long time for fault detection. In long-distance power transmission, the resistance of the wire is used to find the distance of a short circuit fault in the transmission line from the power station.
In this approach, current during the fault is measured which is used to identify the resistance.
As the resistance of the wire per unit length is known, distance to the fault is computed by dividing the resistance of the fault by the unit length resistance.
The resistance of the wire is also used to calculate the circuit breaker capacity in transmission lines.

Voltage, which is more easily measured at the appliance level, can provide a lot of insight into the state of the electrical system.
Plug-level voltage can be sensed to estimate the load on the grid or the distribution transformer \cite{ganuNPlugSmartPlug2012} to identify and reduce the peak load on the feeder.
Frequency variation in the system can be measured and used to estimate load balance on the grid. The system frequency can be used to reduce the load \cite{haoAncillaryServiceGrid2014} and manage the demand when supply is low.
Both of the above approaches utilize the absolute values of these parameters to take action. In contrast, our approach is to use the differences in the value of voltage which enables us to identify appliance level operation.

%rg
Phase connection of the meters is required for efficient electric distribution management, in our case, the same is required to compute the drop in voltage. The voltage measured by the meters is used to identify the correct phase connection of the meters but because of the low frequency of the data, the differences among the phases are very little. Therefore, machine learning techniques are used by \cite{wangPhaseIdentificationElectric2016}, \cite{mitraVoltageCorrelationsSmart2015} to cluster the meters connected to the same phase. We have also used correlation coefficients among the voltage. Time series data is used to identify the correct phase of the meter which is similar to the technique used by \cite{shortAdvancedMeteringPhase2013}. The voltage of the three phases changes in harmony when observed at low frequency, but when observed at the one-second interval, we show that sharp sudden changes in the voltage enable us to detect the correct phase more accurately.
%rg end
Our approach to disaggregation using voltage measurements is complementary to the existing techniques that track appliance operations. Therefore the approach presented in this paper can coexist with all such techniques to improve their performance.
