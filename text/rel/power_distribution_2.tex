When the current wave is in the same phase as the voltage, total power drawn is delivered to the load this power is termed as active power and load that draws this type of power is termed as the active load. Similarly, the power drawn by the inductor or the capacitor which is pushed back to the system by the load is termed as the reactive power and the load drawing such load is term as reactive load. The active and reactive power of the load are represented as the vectors, where the x-axis represents the reactive power and the y-axis represents the active power. The vector sum of these two powers is the total power flowing to the load. this power is called the apparent power as shown in the figure (PQ). Apparent power $P_{apparent}$ can also be computed as follows 

$$P_{apparent} = V_{RMS} * I_{RMS} $$

Where $V_{RMS}$ is root mean square (RMS) value of the voltage and $I_{RMS}$ is the RMS value of the current drawn by the load.

Most of the conventional appliance has both resistive and inductive components therefore they draw both the active and reactive power during their operation. 