
The conventional appliances like heater, incandescent bulb, electric motors have only resistive and inductive components. The current drawn by these appliances is the sine wave. Many electrical motors draw the current of the higher-order frequency. This enabled the identification of such appliances by measuring the total harmonic distortion (THD) which is the ratio of the magnitude of the fundamental frequency to the divided by the amplitude of the current. Figure \ref{fig:phase_harmonic}. It is not possible to distinguish among the appliance that has different harmonics using only the active and reactive power. A smart meter that provides THD of the load are higher-end costly industrial smart meters could not be installed in the common household because of their cost constraints. 

While THD is an important parameter, it does not provide information about the higher-order frequencies present in the current. New appliances that consisted of electronic components draw complicated current wave as shown in the figure \ref{fig:phase_dc} which makes the use of THD less efficient therefore researchers shifted their focus toward collecting high-frequency data to construct signatures. 


A huge variation of current waveforms are observed from appliance to appliance but it is difficult to collect and utilize these signatures because of the following reasons  


\begin{itemize}
    \item High volume of data: To sense and analyze the entire current waveform a higher data sampling rate is required which is in the range of 10-50 kHz collecting and sending the data to the server at this rate is processing it is difficult on the currently available ES485 type of communication channel.

\item Decomposing the appliance level signature: The meter installed at the mains will sense the waveform generated by combining all the waveforms from the ON appliance at any given time. The task of breaking down this into each component is challenging and computationally very demanding.

\end{itemize}

