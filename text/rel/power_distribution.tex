Electrical power is generated in bulk at large power plants and transmitted to load centers through the high voltage transmission cables. 
The power is supplied through the alternating current induced by the voltage that varies as a sine wave. The current drawn by the load is also alternating but the pattern of the waveform is dependent on the electrical nature of the load. If the load is resistive, the current is also a sine wave and the sine wave coincides with the voltage wave as shown in the figure (res curr). The behavior of a capacitive load is also a sine wave but it leads by an angle of 

%$\pi/2$

if the voltage is given by 

$$v(t) = v_{m}\sin\left(t\right)$$

There $v(t)$ is the instantaneous voltage $t$ is the time and $v_{m}$ is the magnitude of the voltage.
The current drawn by the capacitive load is given by 

$$i_c(t) = c_{m}\sin\left(t+\ \frac{\pi}{2}\right)$$

where $c_{m}$ is the magnitude of the current waveform.

Similarly, the current drawn by an inductive load is also sine wave but it legs the voltage by $\pi/2$ angle. which is given by 

$$i_l(t) = l_{m}\sin\left(t-\ \frac{\pi}{2}\right)$$
 
The power delivered to the appliance is the instantaneous product of the voltage and the current. The voltage and current waves are aligned for the resistive load therefore the power delivered is maximum. Figure (p curve r) shows that the power drawn is always positive. But In the case of capacitive or inductive load, the current and voltage are out of phase by a right angle, therefore the power delivered is zero. 
Figure (p curve c) shows that the power drawn is both positive and negative therefore the net flow of power for the complete voltage cycle is zero.

