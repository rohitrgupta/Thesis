When smart meters are installed for energy measurement, some of the minor aspects of the installation are ignored. One such aspect is the connection polarity of the three phases. For some meters in our setup, the polarity was not the same as the mains meter. This difference in phase connection does not cause any issue while measuring energy consumption but turns out to be critical for comparing phase voltage, since comparing the voltage of two different phases will give wrong estimates. We understand that this is a common problem and will be more prevalent when single phase meters are installed. The variation in the voltage sensed at such a meter can be used to identify the correct phase connection for the meter.

When we install solar roof-tops both in residences and in larger buildings, we need to ensure that  a panel is connected to the right phase. In fact, most building managers have an incorrect circuit map of the building which can be quite large \& complex in commercial buildings. So identifying phase at every point and socket correctly is an important aspect of correcting the circuit map of a large building. Since buildings are also commercially billed, they are penalized for imbalanced phases by the utility that one cannot address without identifying the phase of different loads in the building.

When the voltage is observed it varies in a narrow range, and all the three phases appear to change in harmony when observed for a long time interval, but for a very short duration, the variation is distinct for each phase. This can be observed in the figure \ref{fig:voltageChanges} where the voltage for the three phases sensed by the mains meter changes differently. This difference in variation can be used to identify the correct phase of the meter for which the phase is not known.
%rg
%Voltage sample for one minute (60 samples) are take from the meter with the unknown phase $(v_x)$ and the reference meter $(v_1, v_2, v_3)$ are taken.
%The correlation coefficient for all the combination of voltages for the reference meter are computed. If all the correlation coefficients are lower than a threshold, correlation coefficient of the voltages of the reference meter with the unknown meter are measured. If the correlation coefficient for the reference meter are not lower than the threshold the process is repeated for the next one minute interval.
%A one minute sample of voltage, from each phase of the mains meter, is used as a reference for this purpose. The voltage series for each phase is collected given by $v_1$,  $v_2$, and $v_3$.  Correlation coefficients are computed for all the voltage pairs. If all the correlation coefficients are lower than a threshold, a voltage sample for the same duration for the unknown meter is taken, and its correlation coefficients with the three voltages are computed.
%The meter is assigned the phase for which the  correlation coefficient is highest. In the figure \ref{fig:Freq_change}, the voltage observed at meter \textit{with Unknown phase} is also shown. The correlation coefficient of this meter is highest with $v_1$ as shown in the table \ref{tab:corr_coef} which means that this meter is connected with phase 1.

We take the time series sample of the voltage at the reference meter and make sure that there is sufficient difference in the pattern for the three phases. This is made sure by computing the correlation coefficients for three phases. Should any of the correlation coefficients be more than a threshold value, the sample is discarded and the next sample is taken. We utilized samples of 6-minute duration and the threshold of 0.85 for this purpose. Approximately 40\% of the samples are discarded because of this criterion. In such a case, the sample starting time is increased by 30 seconds and the process repeated with the new sample.

The voltage sample at the same time is used to compute its correlation coefficients with the three phases of the reference meter. If the highest correlation coefficient is higher than another threshold (0.95), its phase is assigned to the unknown phase else the process is started from the beginning. The worst accuracy obtained by this method is 99.83\%. Because of the discarding of the samples, the time required to identify the phase varies. In 88\% of the cases, the phase was identified in one attempt. In our scenario, during the night time when there is less variation in the voltage, the algorithm has lower chances of identifying the phase voltage in one attempt.
%rg end
Phase correction is a one time task that is performed for all the meters once and the results obtained are saved in a mapping table which is used for translating the phase name while using the meter data.
