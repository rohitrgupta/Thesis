%\begin{figure}
%\centering
%\includegraphics[width=1\linewidth]{images/G3_M1_20140107_12}
%\caption{Current spike are generated when an appliance is switched on}
%\label{fig:G3_M1_20140107_12}
%\end{figure}

Spikes are very regular phenomenon on a power system, they occur whenever an inductive  appliance is switched on, Figure \ref{fig:G3_M1_20140107_12} shows such a spike caused by starting an AC. These spikes are generated because of high starting torque required by the induction motors. Normally this startup current could be up to 2-3 times the normal running current of the appliance. Motor consume a constant current once it is started and running.  

When such appliance fail to start, the starting high current is drawn but the constant current after the initial startup phase is not drawn. This failure to start can be observed as a current or power spike as shown in the figure \ref{fig:regularSpike}. 

Detecting such spikes is a challenging task because of following reasons 
\begin{itemize}
	\item Too many spikes: Spikes are very common in the a electrical system therefore it is difficult to find the spike caused by faulty appliance.
	\item Too many appliance: Large number of appliance are connected to the system which makes the isolation of the faulty system vary difficult.
\end{itemize}

