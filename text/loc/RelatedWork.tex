A good number of energy meters are designed by researchers to serve different objectives, like plug monitoring, ease of installation, energy harvesting or controlling the attached appliances but none have focused on providing all the data required for the objective of energy disaggregation.

PowerBlade \cite{debruinPowerBladeLowProfileTruePower2015} defined and performed the tasks of reducing the AC mains voltage and current for the purpose of measuring, calculating power and communicating the same. Our meter extends tasks of the smart meter by communicating the high-frequency values of current and voltage required by many disaggregation algorithms \cite{hassanEmpiricalInvestigationVI2014} \cite{guptaElectriSenseSinglepointSensing2010} \cite{gulatiDepthStudyUsing2014} \cite{BOLT} and communicating only when a significant change in the values is observed.

Kill A Watt \cite{KillWattMeter}, and many similar plug level meters are available in the market. These meters are only suitable to identify the power consumption of appliances that can be plugged into the sockets. They are not good for continuous monitoring of the appliance or measuring the power consumption of the appliances that are directly connected through the switchboard.

There are some energy meters like \cite{mogheDesignLowCost2010}, \cite{caiSelfpoweredSmartMeter2012}, \cite{debruinMonjoloEnergyharvestingEnergy2013}, and \cite{villaniContactlessThreephaseAutonomous2016} which focus on the ease of installation and use energy harvesting to run the meter. The limitations of these meters are the nonavailability of energy to harvest when the appliance is off, they have to compromise on accuracy also because of energy constraints. IoT based smart energy metering system \cite{al-aliIoTBasedSmartUtility2018} \cite{yaghmaeeDesignImplementationInternet2018} focuses on sensing other parameters or collecting water or gas consumption data together or collecting temperature humidity, and air quality data. The act as the data collection agents which take the responsibility of sending the collected data to the servers.

Visualization platforms like \cite{luDesignImplementationPower2016} focus on using the existing architecture of the power system for information acquisition and display it to the user to provide better insight into the energy consumption pattern and reducing the power consumption. Our meter will enhance the capabilities of such visualization platforms by enabling them to provide a better understanding of the appliance being used. As our meter can be used without additional communication hardware and require less communication bandwidth, energy visualization platforms can use such services efficiently at a lower cost.

REDD \cite{kolterREDDPublicData2011} is a very popular home energy consumption data set. It provides continuous low-frequency power consumption data and high-frequency current and voltage data when a transition in power consumption is observed. The technique used by REDD is very efficient in reducing the storage requirement for the data set. Out meter also utilizes a similar technique to reduce the storage for sending the power readings from the meter.

Studies have suggested that the biggest obstacle is to identify the cost bearer for the new meters \cite{Romer2012}. A cheaper meter that can be upgraded will help in overcoming such obstacles. Once installed, a smart meter can become a proxy for the water and gas meter \cite{Zivic2015} too. When Smart meters become omnipresent, they will be performing multiple tasks including sensing other environmental parameters and controlling appliances \cite{tripathyPrepaidEnergyMetering2010} \cite{ganuNPlugSmartPlug2012} and identifying faulty appliances \cite{palaniPuttingSmartMeters2014}.
