

%Spikes which are associated with appliance switching should not be considered as fault. Power consumption in the system is generally steady with transients for starting or stopping the appliance. When we pass the data from a median filter, spikes get removed from the data without impacting the load pattern of the data as shown in the figure. used to identify Spikes without appliance switching as follows. 


%# Algorithm for Spike identification
Spikes are very regular phenomenon in a power system, they occurs whenever a appliance is switched on. %Spikes that are associated with appliance switching should not be considered as fault. Faults are the spike after which there is no appliance load \ref{fig:regularSpike}.  
Power consumption at the building level meter can be used to find spike in the system.

Vector $\vec{p}$ is time series of the power consumption samples collected periodically between time $0$ and  $n$. 

%$$\vec{p} = \left \{  p(t) : t = 0 ... t_{max}    \right \}$$
$$\vec{p} = [ p_0, p_1, p_3 ... p_n  ]^T $$

AS the power consumption is generally steady, it can be passed through a median filter $medfit_m$ to obtain a spike free power consumption data. Width of median filter is $2 m + 1$ which is a function of spike width $m$ to be removed. 

$$ \vec{q} = medfit_m(\vec{p}) $$

difference of $\vec{p}$ and $\vec{q}$ can provide all the spikes in the system, but these spikes will be because of both faulty and non faulty appliances. If the spike are not followed by change in load, resultant $\vec{q}$ will be flat. Times when these spike are present can be identified by comparing the  positive transition in both $\vec{p}$ and $\vec{q}$. 

$$T_+ = \left \{ t :  p_t >  p_{t-1} + p_{threshold}   \right \}$$

$$T_a = \left \{ t :  q_t >  q_{t-1} + q_{threshold}  \right \}$$

$T_a$ are the time instances when some appliance is switched on and $T_+$ are the time instances when a positive transition is observed in $\vec{p}$. $p_{threshold}$ and $q_{threshold}$ can be chosen to based type of the appliances and magnitude of spike generated by these appliances. If a time instance is present in $T_+$ and not present in $T_a$ it is a spike probably not associated with appliance switching.

$$ T_s = \left \{ t :  t \in T_+,  t \notin T_a \right \} $$  

Based on the values of $p_{threshold}$ and $q_{threshold}$ we may get excess number of spikes or fewer number of spike. Number of false positive spike increases when large number of appliances and a verity of appliance are involved. 

To identify the faulty AC from these large number of false positive spikes, cyclic nature of the spike should be identified. The time interval between the consecutive spikes are identified by plotting histogram of the time interval between spikes. A unusual high value is observed in the plot for a time period equal to the periodicity of the faulty appliance. Our objective is to identify the appliance that caused the spikes at these time instance.

This can be used to identify the time instance when the spike is observed. The meter installed at building level is sufficient to identify presence of spike in the data. 
