An induction motor is used to compress the coolant in the AC. Amount of work required to compress the coolant increases with the quantity of already compressed coolant. If motor is started when the coolant is already compressed, torque required is too high which causes even higher starting current that can damage the motor. To avoid the risk of damage, Air Conditioners are prevented from starting when the coolant is compressed. Simple technique to do this it to allow the coolant to decompress before starting the motor. It is assumed that when the motor stops, coolant is maximum compressed and time required to decompress the coolant to a favorable level is computed, which can be termed as conditioning delay. Once the motor is stopped, it is allowed to start after only after conditioning delay. In case of failure, timer is initiated and the Air Conditioner attempts to start again after conditioning delay. In a faulty Air Conditioner this cycle is repeated continuously and is observed as repeated spike of power or current.

To find periodic spike, time interval between the consecutive spikes are computed and a histogram of the time interval is plotted. The frequency of time interval follows an inverse square curve as the spike occurrence rate is suppose to be a normal distribution. On this curve a peak is visible which indicates an unusually high rate of spike recurrence after 355 second. The high frequency value is for a time period equal to the periodicity of the faulty appliance. Presence of this peak indicates that there is faulty appliance in the system. Once the presence of faulty appliance is confirmed, next objective is to identify the location of this faulty appliance.

