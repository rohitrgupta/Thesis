In an electrical distribution system, some power is always lost in different stages of delivery.
%Losses can be minimized either by decreasing the wire resistance or by reducing the current.
It is possible to deliver the same power to an appliance by decreasing the current and increasing the voltage by the same factor. This approach is utilized in the long-distance transmission lines where power is transmitted at a very high voltage to minimize these losses. But, a high voltage is not suitable to be used in human proximity because of it's hazardous nature. The only option available is to reduce the power loss is to reduce the resistance of the wire by increasing the diameter.
%, but it can not be done beyond a certain level because of cost and other practical considerations. The lower resistance of the wire can be achieved by increasing the thickness of the wire.
But it can not be done beyond a limit because of the cost consideration. The resistance of the wire can be reduced up to a level beyond which the power loss is not significant enough to justify the increase in the wire cost. In practice, this loss can be in the range of 4-6 \% of the peak load capacity of the line \cite{ElectricalWireGauges}. This can be directly translated to a 4-6 \% drop in the voltage observed at the appliance when operating at its full load.

The second impact of cost consideration is the sharing of a supply line across multiple appliances; this reduces both the cost of installation and power losses. In any common home installation, the power is supplied from the feeder to the house distribution point through a single wire, from where it is connected to different switchboards, from the switchboard a single wire is connected to individual appliances. Such electrical distribution can be represented by a tree structure as in Figure \ref{fig:distribution} where the individual appliances are at the leaf level and the feeder is at the root level.
We will use this as our running example for the rest of the paper.
